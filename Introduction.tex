\chapter{Introduction}

Until recently, the sensing and localization capabilities of small unmanned aerial vehicles (UAVs) have been computationally constrained due to restrictions on both the size and weight of sensors and onboard computer systems. Recent advances in miniaturized desktop computers, as well as low-power Graphics Processing Units (GPUs), have enabled a new class of computationally intensive algorithms to run onboard small aircraft without degrading flight time or other performance metrics. Specifically, small drones now present a viable platform for high-fidelity Simultaneous Localization and Mapping (SLAM), visual object recognition, and state estimation algorithms employing numerous heterogeneous sensors. In addition, the rising popularity of drones among hobbyists and researchers has fueled tremendous growth in the markets for brushless motors, electronic speed controllers, and airframes. With high-quality flight hardware and lightweight computers readily available, the arsenal of the robotics researcher has never been better stocked. The work detailed in this thesis is, in many ways, the product of these advantageous market conditions. With small aircraft capable of lifting more payload and computers now lighter and smaller in footprint, the environment is ripe for outfitting high-performance vehicles with top-notch sensing capabilities.

This work centers on an algorithm known as the Unscented Kalman Filter (UKF), a sensor fusion method for estimating the state of systems such as small aircraft. For years, this algorithm was inaccessible to roboticists and aerospace researchers due to its computational complexity. Computers capable of running the algorithm were simply too large, too heavy, and too power-hungry to be viable onboard components of small UAVs. The advent of smaller computers has changed this state of affairs. In this thesis, we present a formulation of the Unscented Kalman Filter aimed at estimating the pose of a quadcopter. This UKF formulation fuses outputs from only two sensors: an inertial measurement unit (IMU) and a downward-facing monocular camera. These two data streams are fused to estimate the vehicle's pose in real time. We propose this UKF framework as a hardware-minimal starting point for advanced multi-sensor fusion in UAV navigation.

\section{Personal Motivation}

\textit{This section is largely comprised of background information regarding certain experiences leading up to, and motivating, the research developed in my thesis. This section does not contain any technical material, and is included only to provide context to the larger work.}

\vspace{1em}

\begin{wrapfigure}{r}{0.4\textwidth}
  \centering
    \includegraphics[width=0.4\textwidth]{CARD_gearREALLYBIG}
  \caption[CARD Team Logo]{The CARD team logo.}
  \label{fig:card_logo}
\end{wrapfigure}

I first took an interest in unmanned aircraft in the fall of 2012, my sophomore year of college. In search of an exciting engineering challenge, several of my friends and I founded the Cooperative Autonomous Robotics Design (CARD) team at Virginia Tech. Our core team consisted of a dozen students devoted to designing and competing with drones and other robotic vehicles. Our team, guided by my future graduate adviser, Dr.~Kevin Kochersberger, entered two design competitions and brought home two awards for the university. We were also one of four university robotics teams selected by the Smithsonian Institute to participate in the opening ceremony for Robotics Week~2011. These early experiences with the team brought me into contact with new skills such as microcontroller programming, Proportional-Integral-Derivative (PID) controller design, basic mechatronics, and computer-aided design (CAD) modeling.

After two years of involvement with the CARD team, I applied for an internship at the National Institute of Aerospace\footnote{\url{http://www.nianet.org}} (NIA) in Hampton, Virginia. In the summer of 2014, I was part of a team of NIA researchers working on the Flying Donkey Challenge\footnote{\url{http://www.flyingdonkey.org}}, an international engineering competition centered around the idea of ``flying donkeys,'' full-sized autonomous airplanes capable of quickly carrying cargo between small airports in rural Africa. This competition, unfortunately now defunct, was divided into a number of sub-challenges focusing on different technical objectives such as precision landing and collision avoidance. Our team's goal was to design an inexpensive navigation system that could reliably guide unmanned aircraft during a Global Positioning System (GPS) blackout. This project introduced me to many of the technologies and techniques that would later become my major research interests, particularly the Robot Operating System\footnote{\url{http://wiki.ros.org}} (ROS), Kalman Filtering, and sensor fusion.

\begin{wrapfigure}{l}{0.4\textwidth}
  \centering
    \includegraphics[width=0.4\textwidth]{april_tag}
  \caption[Example April Tag]{An example of an April tag.}
  \label{fig:april_tag}
\end{wrapfigure}

Through my internship at the NIA, I met Dr.~Danette Allen, head of the NASA Langley Autonomy Incubator. During my 2014--15 academic year, Dr.~Allen sponsored the CARD team to design and build two autonomous multirotor delivery drones. These aircraft were capable of delivering 5\nobreakdash-lb packages to distances of up to 2.5~miles (or 5~miles, round trip). In addition, these vehicles were able to land precisely on 1~m$^2$ April tags such as that found in Figure~\ref{fig:april_tag}\footnote{\url{http://wiki.ros.org/apriltags\_ros}}. Following the completion of this project, I worked as a summer intern at the Autonomy Incubator, thereby further advancing my interest in Kalman filtering, sensor fusion, visual localization.

During the summer of 2015, I began the research that evolved into my thesis project, studying Visual-Inertial Navigation (VIN) and the Unscented Kalman Filter (UKF). As I read more and more on these subjects, I became interested in the design of the algorithm underlying the UKF. Unlike many other formulations of the Kalman Filter, the UKF has a notably limited dependence on information about the system under scrutiny (this \textit{system agnosticism} is discussed in more detail in Chapter~\ref{ch:Alg_Design}). The more I learned about the UKF, the more I became excited about the idea of taking advantage of this limited dependence trait to build a minimalistic software interface by which a wide variety of disparate systems could be tracked and studied in a ROS framework. I envisioned a ``one-stop shopping'' experience for massively reusable and customizable filtering profiles that could fulfill the needs of researchers and roboticists who may have little knowledge of state estimation techniques. This vision eventually drove my development of the \texttt{kalman\_sense} ROS package, cementing my interest in unmanned aerial vehicle (UAV) state estimation processes and controls.


\section{Organization of this Document}

\subsection*{Prior Work}

In Prior Work, we explore recent contributions to loosely coupled filter-based navigation and state estimation processes. We focus primarily on a number of impactful publications coming from the Autonomous Systems Lab\footnote{\url{www.asl.ethz.ch}} (ASL) at ETH~Zurich\footnote{\textit{Eidgen{\"o}ssische Technische Hochschule Z{\"u}rich}, the Swiss Federal Institute of Technology in Zurich.} and the University of Pennsylvania's GRASP Lab\footnote{\url{www.grasp.upenn.edu}}. We define the current state of the art in filter-based navigation and establish the research context in which this thesis exists.

\subsection*{Algorithm Design and Implementation}

Because of the algorithmic nature of state estimation processes, we explore in detail the design and implementation of the \texttt{kalman\_sense} ROS package. We discuss plant model abstraction as well as code organization and data flow and then summarize the process by which one could extend \texttt{kalman\_sense}'s functionality and the advantages of system-agnostic algorithm design.

\subsection*{Experimental Design}

In this section, we first establish the goals of the testing regimen and then discuss the real-world execution of these goals. We discuss important statistical methods for characterizing the system's effectiveness as well as data collection procedures and post-processing. The system's physical testing infrastructure is explored in detail.

\subsection*{Experimental Results}

In Experimental Results, we evaluate the system's performance during testing and seek out any limiting factors that influence estimation accuracy. We probe for possible improvements to the algorithm and provide a notional understanding of the system's theoretical effectiveness in real-world scenarios.

\subsection*{Conclusions}

We summarize the contributions made in this thesis, the effectiveness of the \texttt{kalman\_sense} ROS package, and the insights acquired during programming and testing. 

\subsection*{Future Work}

In Future Work, we expand upon the possible improvements proposed in Experimental Results and also offer a number of applications for the algorithms and processes developed herein. Specific examples of heterogeneous fleet management and Unmanned Traffic Management (UTM) are explored.